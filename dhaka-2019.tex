\documentclass{article}
\usepackage{amsmath}
\usepackage{amsfonts}
\usepackage{mathtools}
\usepackage{gensymb}
\usepackage[a4paper,inner=1.5cm,outer=1cm,top=2cm,bottom=0.5cm]{geometry} 
\usepackage{xcolor}
\usepackage{cancel}
\usepackage{tikz}
\usepackage{pgfplots}
\usepackage{multicol}
\setlength{\columnsep}{1cm}
\usetikzlibrary{intersections}
\usetikzlibrary{intersections,calc,angles,quotes}
\usetikzlibrary{calc,angles,positioning,intersections,quotes,decorations.markings}
\usepackage{tkz-euclide}
\usetikzlibrary{backgrounds}
\usetikzlibrary{calc,through}
\usetikzlibrary{angles}
\usetikzlibrary{fadings}
\usetikzlibrary{shapes.geometric}
\usetikzlibrary{shapes.symbols}
\usepackage{draftwatermark}
\usepackage{mathptmx}

\SetWatermarkText{Mathema Shukur}
\SetWatermarkFontSize{2 cm}
\usepackage[utf8]{inputenc}
\usepackage{fontspec}

\setmainfont{[Kalpurush.ttf]}
\newfontface{\en}{[Arial.ttf]} %%this is optional, if you want to use a secondary font. Any english font is supported
\usepackage{tikz}
\usetikzlibrary{arrows}

\begin{document} 
(১) $A(-4,0)$এবংB(0,-3) বিন্দুদ্বয়ের সংযোগ রেখার ঢাল,\\
$=\frac{-3-0}{0+4}$\\
 $=-\frac{3}{4}$\\
 (২) $AB$ রেখার সমীকরণ,\\
 \begin{align*}
 	\frac{x+4}{-4-0}&=\frac{y-0}{0+3}\\
 	\frac{x+4}{-4}&=\frac{y}{3}\\
 	3x+12&=-4y\\
 	3x+4y+12&=0\\
 \end{align*}
 মূলবিন্দু$(0,0)$ থেকে$3x+4y+12=0$ রেখার লম্ব দূরত্ব ,\\
 $\frac{|3(0)+4(0)+12|}{\sqrt{3^2+4^2}}$\\
 $\frac{12}{5}$\\
(৩)$(3,\sqrt{3})$ বিন্দুর পোলার স্থানাঙ্ক নির্ণয় কর।\\
$x=3$\\
$y=\sqrt{3}$\\
\begin{align*}
	r&=\sqrt{x^2+y^2}\\
	r&=\sqrt{(3)^2+(\sqrt{3}^2)}\\
	r&=\sqrt{9+3}\\
	r&=\sqrt{12}\\
	r&=2\sqrt{3}\\
\end{align*}
 বিন্দুটি প্রথম চথুর্ভাগে অবস্থিত,\\
 \begin{align*}
 	\theta&=\tan^{-1}\frac{y}{x}\\
 	\theta&=\tan^{-1}\frac{\sqrt{3}}{3}\\
 	\theta&=\tan^{-1}\frac{1}{\sqrt{3}}\\
 	\theta&=\frac{\pi}{6}\\
 \end{align*}
 পোলার স্থানাঙ্ক, $(2\sqrt{3},\frac{\pi}{6})$\\
(৪)$PA$ রেখার সমীকরণ,\\
$\frac{x-3}{3-3}=\frac{y-0}{0-\sqrt{3}}$\\
$x-3=0$\\
$AB$ রেখার সমীকরণ,\\
$\frac{x}{3}+\frac{y}{2}=1$\\
$2x+3y=6$\\
$PA$ ও $AB$ রেখার অন্তর্ভুক্ত কোণ সমূহের সমদ্বিখণ্ডক রেখাসমূহের সমীকরণ,\\
\begin{align*}
	\frac{x-3}{\sqrt{1^2+0^2}}&=\pm\frac{2x+3y-6}{\sqrt{2^2+3^2}}\\
	x-3&=\pm\frac{2x+3y-6}{\sqrt{13}}\\
\end{align*}
(+) চিহ্ন  নিয়ে পাই,\\
$\sqrt{13(x-3)}=2x+3y-6$\\
$(\sqrt{13}-2)x-3y-3\sqrt{3}+6=0$\\
(-) চিহ্ন নিয়ে পাই,\\
$(\sqrt{13}+2)x+3y-3\sqrt{3}-6=0$\\
\end{document}