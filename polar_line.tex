\documentclass{article}
\usepackage[a4paper,inner=1.5cm,outer=1.5cm,top=2cm,bottom=0.5cm]{geometry} 
\newcommand{\myhash}{\raisebox{\depth}{\#}}
\usepackage{multicol}
\usepackage{tikz}
\usepackage{amsmath}
\usepackage{mathtools}
\usepackage{gensymb}
\usepackage{pgfplots}
\usepackage{cancel}
\usetikzlibrary{intersections}
\usetikzlibrary{intersections,calc,angles,quotes}
\usetikzlibrary{calc,angles,positioning,intersections,quotes,decorations.markings}
\usepackage{tkz-euclide}
\usetikzlibrary{backgrounds}
\usetikzlibrary{calc,through}
\usetikzlibrary{angles}
\usetikzlibrary{fadings}
\usetikzlibrary{shapes.geometric}
\usepackage{draftwatermark}
\usepackage{mathptmx} 
\SetWatermarkText{Mathema Shukur}
\SetWatermarkFontSize{3 cm}
\usepackage[utf8]{inputenc}
\usepackage{fontspec}
\setlength{\columnsep}{2cm}
\setmainfont{[Kalpurush.ttf]}
\newfontface{\en}{[Arial.ttf]}
\usetikzlibrary{shapes.geometric, arrows}
\tikzstyle{startstop} = [draw, ellipse,minimum width=3cm, minimum height=1cm,text centered, draw=black]
\tikzstyle{io} = [trapezium, trapezium left angle=70, trapezium right angle=110, minimum width=3cm, minimum height=1cm, text centered, draw=black]
\tikzstyle{process} = [rectangle, minimum width=3cm, minimum height=1cm, text centered, draw=black]
\tikzstyle{decision} = [diamond, minimum width=3cm, minimum height=1cm, text centered, draw=black]
\tikzstyle{connector} = [circle, radius=1cm, text centered, draw=black]
\tikzstyle{arrow} = [thick,->,>=stealth]
\tikzstyle{line} = [draw, -latex']  
\begin{document} 
	\Large
	যাদের জন্যে প্রযোজ্যঃ  	\textcolor{black}{একাদশ ও দ্বাদশ শ্রেণীর শিক্ষার্থী} \\
	বিষয়ঃ \textcolor{black}{উচ্চতর গণিত ১ম পত্র} \\
	অধ্যায়ঃ \textcolor{black}{৩-সরলরেখা }\\ 
	Subtopicঃ \textcolor{black}{কার্তেসীয় ও পোলার স্থানাঙ্ক }\\	
	(1) রাজশাহী বোর্ড-২০১৯\\
	$x+3y+3=0$ রেখাটি y- অক্ষকে যে বিন্দুতে ছেদ করে তাঁর পোলার স্থানাঙ্ক নির্ণয় কর। \\
	\\
	(2) যশোর বোর্ড-২০১৯\\
	কোনো বিন্দুর পোলার স্থানাঙ্ক $(5,90\degree)$ হলে কার্তেসীয় স্থানাঙ্ক নির্ণয় কর। \\
	\\
	(3) যশোর বোর্ড-২০১৯; সকল বোর্ড-২০১৮;ঢাকা বোর্ড-২০১৭;দিনাজপুর বোর্ড-২০১৭\\
	$(-1,\sqrt{3})$ এর পোলার স্থানাঙ্ক নির্ণয় কর। \\
	\\
	(4) বরিশাল বোর্ড-২০১৯\\
	$(-1,-1)$ এর পোলার স্থানাঙ্ক নির্ণয় কর।\\
	\\
	(5) ঢাকা বোর্ড-২০১৯\\
	$(3,\sqrt{3})$ এর পোলার স্থানাঙ্ক নির্ণয় কর। \\
	\\
	(6) সকল বোর্ড-২০১৮\\
	$(-4,-4)$ এর পোলার স্থানাঙ্ক নির্ণয় কর। \\
	\\
	 (7) বরিশাল বোর্ড-২০১৭\\
	কোনো বিন্দুর পোলার স্থানাংকের কোটি $90\degree$ হলে ঐ বিন্দুর কার্তেসীয় স্থানাংকের ভুজ নির্ণয় কর।\\
	\\
		(8) সিলেট বোর্ড-২০১৭\\
	$(1,-1)$ এর পোলার স্থানাঙ্ক নির্ণয় কর। \\ 
	\\  
		(9) সিলেট বোর্ড-২০১৭\\
	$(-2,-\sqrt{2})$ এর পোলার স্থানাঙ্ক নির্ণয় কর। \\
	\\   
		(10) কুমিল্লা বোর্ড-২০১৭\\
	$(-\sqrt{2},-\sqrt{2})$ এর পোলার স্থানাঙ্ক নির্ণয় কর। \\   
\end{document}