\documentclass{article}
\usepackage{amsmath}
\usepackage{amsfonts}
\usepackage{mathtools}
\usepackage{gensymb}
\usepackage[a4paper,inner=1.5cm,outer=1cm,top=2cm,bottom=0.5cm]{geometry} 
\usepackage{xcolor}
\usepackage{cancel}
\usepackage{tikz}
\usepackage{pgfplots}
\usepackage{multicol}
\setlength{\columnsep}{1cm}
\usetikzlibrary{intersections}
\usetikzlibrary{intersections,calc,angles,quotes}
\usetikzlibrary{calc,angles,positioning,intersections,quotes,decorations.markings}
\usepackage{tkz-euclide}
\usetikzlibrary{backgrounds}
\usetikzlibrary{calc,through}
\usetikzlibrary{angles}
\usetikzlibrary{fadings}
\usetikzlibrary{shapes.geometric}
\usetikzlibrary{shapes.symbols}
\usepackage{draftwatermark}
\usepackage{mathptmx}

\SetWatermarkText{Mathema Shukur}
\SetWatermarkFontSize{2 cm}
\usepackage[utf8]{inputenc}
\usepackage{fontspec}

\setmainfont{[Kalpurush.ttf]}
\newfontface{\en}{[Arial.ttf]} %%this is optional, if you want to use a secondary font. Any english font is supported
\usepackage{tikz}
\usetikzlibrary{arrows}

\begin{document}
	\Large
	(১) $AB$ এর ঢাল =-$\frac{a}{b}$\\
	=-$\frac{6}{-4}$\\
	=$\frac{3}{2}$\\
	$CD \perp AB$\\
	$CD$এর ঢাল =-$\frac{1}{\frac{3}{2}}$\\
	=-$\frac{2}{3}$\\
(২)$AB$ এর সমীকরণ,\\
\begin{align*}
	6x-4y+24&=0\\
	6x-4y&=-24\\
	\frac{6x}{-24}-\frac{4y}{-24}&=1\\
	\frac{x}{-4}+\frac{y}{6}&=1\\
\end{align*}
$A$ বিন্দুর স্থানাংক $(-4,0)$\\
$B$ বিন্দুর স্থানাংক $(0,6)$\\
$P$ বিন্দুর স্থানাংক,
\begin{align*}
	\left(\frac{x_1+x_2}{2},\frac{y_1+y_2}{2}\right)\\
	\left(\frac{-4+0}{2},\frac{0+6}{2}\right)\\
	(-2,3)\\
\end{align*}
$OP$এর সমীকরণ,
\begin{align*}
	\frac{x-0}{0+2}&=\frac{y-0}{0-3}\\
	\frac{x}{2}&=\frac{y}{-3}\\
	2y&=-3x\\
	3x+2y&=0\\
\end{align*}
(৩) $x+y-2=0$রেখাটির মূলবিন্দু হতে লম্ব দূরত্ব নির্ণয় কর।\\
$(x_1,y_1)$ বিন্দু হতে$(ax+by+c=0)$ রেখার লম্ব দূরত্ব,\\
$=\frac{||ax_1+by_1+c|}{\sqrt{a^2+b^2}}$\\
$(0,0)$ বিন্দু হতে $(x+y-2=0)$ রেখার লম্ব দূরত্ব,\\
$=\frac{|0+0-2}{\sqrt{1^2+1^2}}$\\
$=\frac{2}{\sqrt{2}}$\\
$=\sqrt{2}$\\
(৪)$AB$ এর সমীকরণ,\\
\begin{align*}
	\frac{x-x_1}{x_1-x_2}&=\frac{y-y_1}{y_1-y_2}\\
	\frac{x-2}{2+2}&=\frac{y-1}{1-0}\\
	\frac{x-2}{4}&=y-1\\
	x-2&=4(y-1)\\
	x-2&=4y-4\\
	x-4y+2&=0\\
\end{align*}
$AC$ এর সমীকরণ,\\
\begin{align*}
	\frac{x-x^1}{x^1-x^2}&=\frac{y-y^1}{y^1-y^2}\\
	\frac{x-2}{2+2}&=\frac{y-1}{1+2}\\
	\frac{x-2}{4}&=\frac{y-2}{3}\\
	3x-6&=4y-4\\
	3x-4y-2&=0\\
\end{align*}
$AB$ও $AC$ এর অন্তর্ভুক্ত কোণের সমদ্বিখণ্ডক রেখাদ্বয়ের সমীকরণ,\\
	\begin{align*}
		\frac{x-4y+2}{\sqrt{1^2+(-4)^2}}&=\pm\frac{3x-4y-2}{\sqrt{3^2+(-4)^2}}\\
		\frac{x-4y+2}{\sqrt{17}}&=\pm\frac{3x-4y-2}{5}\\
		5(x-4y+2)&=\pm\sqrt{17}(3x-4y-2)\\
	\end{align*}
(+) চিহ্ন নিয়ে পাই,\\
\begin{align*}
	5(x-4y+2)&=\sqrt{17}(3x-4y-2)\\
	5x-20y+10&=3\sqrt{17}x-4\sqrt{17}y-2\sqrt{17}\\
	(5x-3\sqrt{17})x-(4\sqrt{17}-20)y+10+2\sqrt{17}&=0\\
\end{align*}
(-) চিহ্ন নিয়ে পাই\\
$(5+3\sqrt{17})x-(20-4\sqrt{17})y+(10-2\sqrt{17})=0$\\
\end{document}