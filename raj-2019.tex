\documentclass{article}
\usepackage{amsmath}
\usepackage{amsfonts}
\usepackage{mathtools}
\usepackage{gensymb}
\usepackage[a4paper,inner=1.5cm,outer=1cm,top=2cm,bottom=0.5cm]{geometry} 
\usepackage{xcolor}
\usepackage{cancel}
\usepackage{tikz}
\usepackage{pgfplots}
\usepackage{multicol}
\setlength{\columnsep}{1cm}
\usetikzlibrary{intersections}
\usetikzlibrary{intersections,calc,angles,quotes}
\usetikzlibrary{calc,angles,positioning,intersections,quotes,decorations.markings}
\usepackage{tkz-euclide}
\usetikzlibrary{backgrounds}
\usetikzlibrary{calc,through}
\usetikzlibrary{angles}
\usetikzlibrary{fadings}
\usetikzlibrary{shapes.geometric}
\usetikzlibrary{shapes.symbols}
\usepackage{draftwatermark}
\usepackage{mathptmx}

\SetWatermarkText{Mathema Shukur}
\SetWatermarkFontSize{2 cm}
\usepackage[utf8]{inputenc}
\usepackage{fontspec}

\setmainfont{[Kalpurush.ttf]}
\newfontface{\en}{[Arial.ttf]} %%this is optional, if you want to use a secondary font. Any english font is supported
\usepackage{tikz}
\usetikzlibrary{arrows}

\begin{document}
	\Large
(১) $x-\sqrt{3}y=7$ সরলরেখার ঢাল কত ?
\begin{align*}
	x-\sqrt{3}y&=2\\
	x-7&=\sqrt{3}y\\
	y&=\frac{x-7}{\sqrt{3}}\\
	y&=\frac{1}{3}x-\frac{7}{\sqrt{3}}\\	
\end{align*}
ঢাল $m=\frac{1}{\sqrt{3}}$\\
(২) $x+3y+3=0$ রেখাটি দ্বারা অক্ষদ্বয়ের খন্ডিতাংশের মধ্যবিন্দুর স্থানাংক নির্ণয় কর। 
\begin{align*}
	x+3y+3&=0\\
	x+3y&=-3\\
	\frac{x+3y}{-3}&=\frac{-3}{-3}\\
	\frac{x}{-3}+\frac{3y}{-3}&=1\\
	\frac{x}{-3}+\frac{s}{-1}&=1\\
\end{align*}
$x$ অক্ষের উপর অবস্থিত A(-3,0)\\
$y$ অক্ষের উপর অবস্থিত (0,-1)\\

মধ্যবিন্দুর স্থানাংক $\left(\frac{-3+0}{2},\frac{0-1}{2}\right)$\\
$\left(\frac{-3}{2},\frac{-1}{2}\right)$\\  

(৩) $x+3y+3=0$ রেখাটি y অক্ষকে যে বিন্দুতে ছেদ করে তার পোলার স্থানাংক নির্ণয় কর।
\begin{align*}
	x+3y+3&=0\\
	x+3y&=-3\\
	\frac{x+3y}{-3}&=\frac{-3}{-3}\\
	\frac{x}{-3}+\frac{y}{-1}&=1\\
	\frac{x}{-3}+\frac{y}{-1}&=1\\
\end{align*}
$y$ অক্ষকে $(0,-1)$ বিন্দুতে ছেদ করে। 
\begin{align*}
	x=0, y=-1\\
	r&=\sqrt{x^2+y^2}\\
	r&=\sqrt{0^2+(-1)^2}\\
	r&=\sqrt{0+1}\\
	r&=1\\
\end{align*}
\begin{align*}
	\theta&=\pi+\tan^{-1}|\frac{y}{x}|\\
	\theta&=\pi+\tan^{-1}\\
	\theta&=\pi+\frac{\pi}{2}\\
	\theta&=\frac{3\pi}{2}\\
\end{align*}
(৪)K এর কোন মানের জন্য  $2x-y+7=0$এবং $3x+ky-5=0$ সরলরেখাদ্বয়ের পরস্পর লম্ব।
\begin{align*}
	2x-y+7=0\\
	a_1x_1+b_1y+c_1=0\\
	a_1=2,b_1=-1\\
	3x+Ky-5=0\\
	a_2x+b_2y+c_2=0\\
	a_2=3,b_2=K\\
\end{align*}
লম্ব হওয়ার শর্ত:
\begin{align*}
	a_1a_2+b_1b_2=0\\
	2(3)+(-1)K=0\\
	6-K=0\\K=2\\
\end{align*}
(৫)$2x-3y+c=0$ রেখার উপর দুইটি বিন্দু $P(4,3)$ও$Q(-8,-5)$। $PQ$ রেখাকে $x$ অক্ষ যে অনুপাতে বিভক্ত করে তা বের কর।
ধরি,$PQ$রেখাকে $x$ অক্ষ(a,0) বিন্দুতে$m_1:m_2$ অনুপাতে বিভক্ত করে।
\begin{align*}
	P(4,3), x_1=4, y_1=3\\
	Q(-8,-5) ,x_2=-8 ,y_2=-5\\
	(a,0)&=\left(\frac{m_1x_2+m_2x_1}{m_1+m_2},\frac{m_1y_2+m_2y_1}{m_1+m_2}\right)\\
	\frac{m_1y_2+m_2y_1}{m_1+m_2}&=0\\ 
	\frac{m_1(-1)+m_2(3)}{m_1+m_2}&=0\\
	-5m_1+3m_2&=0\\
	-5m_1&=-3m_2\\
	5m_1&=3m_2\\
	\frac{m_1}{m_2}&=\frac{3}{5}\\
	m_1:m_2&=3:5\\
\end{align*}
$x$অক্ষ $PQ$ রেখাংশকে 3:5 অনুপাতে অন্তবিভক্ত করে।
(৬) $2x-3y+c=0$ রেখার উপর দুইটি বিন্দু P(4,3) ওQ(-8,-5) ।$PQ$ রেখার লম্ব সমদ্বিখণ্ডক দ্বারা $x$ অক্ষের ছেদাংশ নির্ণয় কর।
$PQ$ রেখার সমীকরণ,
\begin{align*}
	\frac{x-x_1}{x_1-x_2}&=\frac{y-y_1}{y_1-y_2}\\
	\frac{x-4}{4+8}&=\frac{y-3}{3+5}\\
	\frac{x-a}{12}&=\frac{y-3}{8}\\
	\frac{x-4}{3}&=\frac{y-3}{2}\\
	2(x-4)&=3(y-3)\\
	2x-8&=3y-9\\
	2x-3y-8+9&=0\\
	2x-3y+1&=0..........(1)\\
\end{align*}
আবার,P(4,3) ও Q(-8,-5) এর মধ্যবিন্দুর স্থানাংক $\left(\frac{4-8}{2},\frac{3-5}{2}\right)$\\
$\left(\frac{-4}{2},\frac{-2}{2}\right)$\\
(-2,-1)\\
(1)নং রেখার লম্বরেখার সমীকরণ,\\
$3x+2y+K=0$..............(2)\\ 
(2)নং রেখাটি(-2,-1) বিন্দুগামী.\\
\begin{align*}
3(-2)+2(-1)+K&=0\\
-6-2+K&=0\\
-8+K&=0\\
K&=0\\
\end{align*}
3x+2y+8=0\\ 
$x$ অক্ষের ছেদাংশ নির্ণয়ে $y=0$ ব্যবহার করে পাই,\\
3x+8=0\\
x=$\frac{-8}{3}$\\
$x$ অক্ষের ছেদাংশ =$\frac{-8}{3}$\\
\end{document}