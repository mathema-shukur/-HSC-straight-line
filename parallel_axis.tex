\documentclass{article}
\usepackage[a4paper,inner=1.5cm,outer=1.5cm,top=2cm,bottom=0.5cm]{geometry} 
\newcommand{\myhash}{\raisebox{\depth}{\#}}
\usepackage{multicol}
\usepackage{tikz}
\usepackage{amsmath}
\usepackage{mathtools}
\usepackage{gensymb}
\usepackage{pgfplots}
\usepackage{cancel}
\usetikzlibrary{intersections}
\usetikzlibrary{intersections,calc,angles,quotes}
\usetikzlibrary{calc,angles,positioning,intersections,quotes,decorations.markings}
\usepackage{tkz-euclide}
\usetikzlibrary{backgrounds}
\usetikzlibrary{calc,through}
\usetikzlibrary{angles}
\usetikzlibrary{fadings}
\usetikzlibrary{shapes.geometric}
\usepackage{draftwatermark}
\usepackage{mathptmx} 
\SetWatermarkText{Mathema Shukur}
\SetWatermarkFontSize{3 cm}
\usepackage[utf8]{inputenc}
\usepackage{fontspec}
\setlength{\columnsep}{2cm}
\setmainfont{[Kalpurush.ttf]}
\newfontface{\en}{[Arial.ttf]}
\usetikzlibrary{shapes.geometric, arrows}
\tikzstyle{startstop} = [draw, ellipse,minimum width=3cm, minimum height=1cm,text centered, draw=black]
\tikzstyle{io} = [trapezium, trapezium left angle=70, trapezium right angle=110, minimum width=3cm, minimum height=1cm, text centered, draw=black]
\tikzstyle{process} = [rectangle, minimum width=3cm, minimum height=1cm, text centered, draw=black]
\tikzstyle{decision} = [diamond, minimum width=3cm, minimum height=1cm, text centered, draw=black]
\tikzstyle{connector} = [circle, radius=1cm, text centered, draw=black]
\tikzstyle{arrow} = [thick,->,>=stealth]
\tikzstyle{line} = [draw, -latex']  
\begin{document} 
	\Large
	যাদের জন্যে প্রযোজ্যঃ  	\textcolor{black}{একাদশ ও দ্বাদশ শ্রেণীর শিক্ষার্থী} \\
বিষয়ঃ \textcolor{black}{উচ্চতর গণিত ১ম পত্র} \\
অধ্যায়ঃ \textcolor{black}{৩-সরলরেখা }\\ 
Subtopicঃ \textcolor{black}{অক্ষের সমান্তরাল সরলরেখা}\\	
\\
(1) রাজশাহী বোর্ড-২০১৯\\
$x-\sqrt{3}y=7$ সরলরেখার ঢাল কত?\\
\\
(2) চট্রগ্রাম বোর্ড-২০১৯\\
$x+y+3=0$ সরলরেখাটি x-  অক্ষের ধনাত্মক দিকের সাথে কত ডিগ্রী কোণ উৎপন্ন করে । \\
\\
(3) বরিশাল বোর্ড-২০১৯\\
$y+x=0$ সরলরেখাটি x-  অক্ষের ধনাত্মক দিকের সাথে কত ডিগ্রী কোণ উৎপন্ন করে । \\
\\
(4) ঢাকা বোর্ড-২০১৯\\
$A(-4,0)$ ও $B(0,-3)$ বিন্দুর সংযোজক রেখার ঢাল নির্ণয় কর।\\
\\
(5) ঢাকা বোর্ড-২০১৭\\
$y=-7x+9$ রেখার সাথে লম্ব রেখার নতি নির্ণয় কর।\\
\\
(6) যশোর বোর্ড-২০১৭\\
$3x+4y+1=0$ সরলরেখার ঢাল কত?\\  
\\
(7) সিলেট বোর্ড-২০১৭\\
$4x-3y+5=0$ সরলরেখার ঢাল কত?\\
\\
(8) ঢাকা বোর্ড-২০১৯\\
$A(3,5)$ ও $B(6,7)$ বিন্দুর সংযোজক রেখার লম্ব দ্বিখন্ডকের ঢাল নির্ণয় কর।\\  
\\
(9) দিনাজপুর  বোর্ড-২০১৭\\
$4x-2y=6$ সরলরেখার ঢাল কত?\\
\\
(10) রাজশাহী  বোর্ড-২০১৭\\
$3x-5y+1=0$ সরলরেখার ঢাল কত?\\
\end{document}