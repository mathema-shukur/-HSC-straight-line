\documentclass{article}
\usepackage{amsmath}
\usepackage{amsfonts}
\usepackage{mathtools}
\usepackage{gensymb}
\usepackage[a4paper,inner=1.5cm,outer=1cm,top=2cm,bottom=0.5cm]{geometry} 
\usepackage{xcolor}
\usepackage{cancel}
\usepackage{tikz}
\usepackage{pgfplots}
\usepackage{multicol}
\setlength{\columnsep}{1cm}
\usetikzlibrary{intersections}
\usetikzlibrary{intersections,calc,angles,quotes}
\usetikzlibrary{calc,angles,positioning,intersections,quotes,decorations.markings}
\usepackage{tkz-euclide}
\usetikzlibrary{backgrounds}
\usetikzlibrary{calc,through}
\usetikzlibrary{angles}
\usetikzlibrary{fadings}
\usetikzlibrary{shapes.geometric}
\usetikzlibrary{shapes.symbols}
\usepackage{draftwatermark}
\usepackage{mathptmx}

\SetWatermarkText{Mathema Shukur}
\SetWatermarkFontSize{2 cm}
\usepackage[utf8]{inputenc}
\usepackage{fontspec}

\setmainfont{[Kalpurush.ttf]}
\newfontface{\en}{[Arial.ttf]} %%this is optional, if you want to use a secondary font. Any english font is supported
\usepackage{tikz}
\usetikzlibrary{arrows}

\begin{document}
	\Large
(১)$ABC$ ত্রিভুজের শীর্ষবিন্দুগুলো $A(2,0)$, $B(5,0)$ও $C(5,4)$ হলে ত্রিভুজের ভরকেন্দ্র নির্ণয় কর।
\begin{align*}
	(x_1,y_2)&=(2,0)\\
	(x_2,y_2)&=(5,0)\\
	(x_3,y_3)&=(5,4)\\
\end{align*}
ভরকেন্দ্রের স্থানাংক, $\left(\frac{x_1+x_2+x_3}{3},\frac{y_1+y_2+y_3}{3}\right)$\\
$\left(\frac{2+5+5}{3}\frac{0+0+4}{3}\right)$\\
$\left(\frac{12}{3},\frac{4}{3}\right)$\\
$\left(4,\frac{4}{3}\right)$\\
(২) মনেকরি,$A$ সরলরেখার ঢাল $m_1$\\
এখন,$m_1$ ঢাল বিশিষ্ট এবং(0,0) বিন্দুগামী রেখার সমীকরণ,\\
$y-0=m_1(x-0)$\\
$y=m_1x$\\ 
আবার,OR রেখার সমীকরণ,\\
$y=-2x$\\
ঢাল$m_2=-2$\\
যেহেতু$AB$ ও$OR$ রেখার মধ্যবর্তী কোণ $45\degree$\\
\begin{align*}
	\tan45\degree&=\pm\frac{m_1-m_2}{1+m_1m_2}\\
	1&=\pm\frac{m_1-(-2)}{1+m_1(-2)}\\
	1&=\pm\frac{m_1+2}{1-2m_1}\\
	(1-2m_1)&=\pm(m_1+2)\\
\end{align*}
(+)চিহ্ন নিয়ে পাই,
\begin{align*}
	1-2m_1&=m_1+2\\
	3m_1&=-1\\
	m_1&=-\frac{1}{3}\\
\end{align*}
(-)চিহ্ন নিয়ে পাই,
\begin{align*}
	1-2m_1&=-(m_1+2)\\
	1-2m_1&=-m_1-2\\
	1+2&=-m_1+2m_1\\
	m_1&=3\\
\end{align*}
(৩) $OPQR$ সামান্তরিকের $OP$, $x$ বরাবর অবস্থিত এবং $OR$ রেখার সমীকরণ ,\\
$y=-2x$\\
যেহেতু, $OP||QR$\\
সুতরাং, $Q(-4,2)$ বিন্দুর কোটি= $R$ বিন্দুর কোটি= $2$\\
অতএব,$R(\alpha,2)$,$y=-2x$ রেখার উপর অবস্থিত.\\
$2=-2\alpha$\\
 $\alpha=-1$\\
$R$ বিন্দুর স্থানাংক,$(-1,2)$\\
$OP$=$QR$=$\sqrt{(-4+1)^2+(2-2)^2}$=$3$\\
$P$ বিন্দুর স্থানাংক, $(-3,0)$\\
$PO$, $x$ অক্ষ বরাবর অবস্থিত এবং$x$ অক্ষের ঋণাত্মক পাশে অবস্থিত।\\
এখন,PR কর্ণের সমীকরণ,\\
\begin{align*}
	\frac{x+3}{-3+1}&=\frac{y-0}{0-2}\\
	\frac{x+3}{-2}&=\frac{y}{-2}\\
	x+3&=y\\
	x-y+3&=0\\
\end{align*}
\end{document}